\documentclass[12pt,a4paper]{article}
\usepackage[utf8]{inputenc}
\usepackage[T1]{fontenc}
\usepackage[bahasa]{babel}
\usepackage{geometry}
\usepackage{longtable}
\usepackage{booktabs}
\usepackage{array}
\usepackage{xcolor}
\usepackage{multirow}
\usepackage{graphicx}
\usepackage{hyperref}
\usepackage{pdflscape}

\geometry{margin=2.5cm}

\title{Tabel Blackbox Testing dan Hasil Penelitian\\Platform EduSorong}
\author{}
\date{}

\begin{document}

\maketitle

\section{Pendahuluan}

Dokumen ini berisi tabel blackbox testing dan hasil penelitian untuk platform crowdfunding EduSorong. Platform ini memungkinkan pengguna untuk membuat kampanye penggalangan dana, melakukan donasi, dan mengelola penarikan dana dengan integrasi payment gateway Midtrans.

\section{Metodologi Testing}

Blackbox testing dilakukan dengan fokus pada fungsionalitas sistem tanpa melihat implementasi internal. Testing dilakukan berdasarkan use case dan requirement yang telah ditetapkan. Setiap test case mencakup:
\begin{itemize}
    \item ID Test Case
    \item Deskripsi
    \item Precondition
    \item Input/Data Test
    \item Expected Output
    \item Actual Output
    \item Status (Pass/Fail)
    \item Keterangan
\end{itemize}

\section{Tabel Blackbox Testing}

\begin{landscape}
\small
\renewcommand{\arraystretch}{1.3}

\begin{longtable}{|
p{0.8cm}|
p{2.2cm}|
p{3.2cm}|
p{4.2cm}|
p{3.2cm}|
p{4.2cm}|
p{3.0cm}|
p{1.6cm}|
}
\caption{Hasil Pengujian Blackbox Platform EduSorong}
\label{tab:blackbox} \\
\hline
\textbf{No} &
\textbf{Fitur} &
\textbf{Skenario Pengujian} &
\textbf{Langkah Pengujian} &
\textbf{Data Uji} &
\textbf{Hasil yang Diharapkan} &
\textbf{Hasil Aktual} &
\textbf{Status} \\
\hline
\endfirsthead

\hline
\textbf{No} &
\textbf{Fitur} &
\textbf{Skenario Pengujian} &
\textbf{Langkah Pengujian} &
\textbf{Data Uji} &
\textbf{Hasil yang Diharapkan} &
\textbf{Hasil Aktual} &
\textbf{Status} \\
\hline
\endhead

\hline
\multicolumn{8}{|r|}{\textit{Dilanjutkan ke halaman berikutnya}} \\
\hline
\endfoot

\hline
\endlastfoot

\multicolumn{8}{|l|}{\textbf{A. Autentikasi Pengguna}} \\ \hline

1 &
Registrasi &
Pengguna baru berhasil mendaftar &
Buka halaman \texttt{/auth}, pilih tab \textit{Register}, isi data valid, klik tombol \textit{Daftar} &
Nama: John Doe; Email: john@example.com; Password: password123; Konfirmasi password sama &
Akun baru tersimpan, pengguna login otomatis dan diarahkan ke halaman utama &
Sistem menyimpan akun baru, membuat sesi login, dan mengarahkan ke beranda &
\textcolor{green}{Berhasil} \\ \hline

2 &
Registrasi &
Registrasi dengan email yang sudah terdaftar ditolak &
Lakukan registrasi menggunakan email yang sudah terdaftar sebelumnya &
Nama: Jane Doe; Email: john@example.com; Password: password123; Konfirmasi sama &
Form menolak input dan menampilkan pesan bahwa email sudah digunakan &
Validasi menolak email duplikat dan menampilkan pesan kesalahan pada field email &
\textcolor{green}{Berhasil} \\ \hline

3 &
Login &
Pengguna berhasil login dengan kredensial valid &
Buka halaman \texttt{/auth}, pilih tab \textit{Login}, isi email dan password yang benar, klik \textit{Masuk} &
Email: john@example.com; Password: password123 &
Pengguna berhasil login, sesi dibuat, dan diarahkan ke halaman yang diminta / beranda &
Sistem mengautentikasi pengguna dan mengarahkan ke beranda tanpa error &
\textcolor{green}{Berhasil} \\ \hline

\multicolumn{8}{|l|}{\textbf{B. Manajemen Kampanye}} \\ \hline

4 &
Buat Kampanye &
Kampanye baru berhasil dibuat dengan data lengkap &
Login, buka \texttt{/dashboard/kampanye/create}, isi semua field wajib dengan data valid, klik \textit{Simpan} &
Judul, lokasi, target dana, ringkasan, deskripsi, dan gambar valid &
Data kampanye tersimpan, kampanye muncul di dashboard pengguna dan daftar kampanye publik &
Kampanye tersimpan di basis data dan tampil di dashboard serta halaman \texttt{/kampanye} &
\textcolor{green}{Berhasil} \\ \hline

5 &
Buat Kampanye &
Validasi judul kampanye wajib diisi &
Login, buka form pembuatan kampanye, kosongkan judul dan isi field lain, lalu simpan &
Judul kosong; target dana: 50.000.000; field lain terisi &
Form menolak penyimpanan dan menampilkan pesan bahwa judul wajib diisi &
Form menampilkan pesan error pada field judul dan kampanye tidak dibuat &
\textcolor{green}{Berhasil} \\ \hline

6 &
Lihat Detail Kampanye &
Detail kampanye tampil lengkap &
Buka halaman detail kampanye melalui \texttt{/kampanye/\{id\}} dari daftar kampanye &
ID kampanye yang sudah ada di sistem &
Halaman menampilkan judul, deskripsi, progress donasi, pengelola, update, dan form donasi &
Detail kampanye tampil lengkap sesuai data di basis data &
\textcolor{green}{Berhasil} \\ \hline

\multicolumn{8}{|l|}{\textbf{C. Donasi dan Pembayaran}} \\ \hline

7 &
Donasi (QRIS) &
Pembuatan transaksi donasi dengan metode QRIS &
Pada halaman detail kampanye, pilih nominal donasi, pilih metode QRIS, lalu konfirmasi donasi &
Nominal: Rp 50.000; Metode: QRIS; Nama donor boleh anonim &
Sistem membuat record payment berstatus \textit{pending} dan menampilkan kode/QR QRIS &
Payment tercatat dengan status \textit{pending} dan halaman QRIS muncul untuk discan &
\textcolor{green}{Berhasil} \\ \hline

8 &
Donasi &
Validasi minimum nominal donasi &
Pada halaman kampanye, masukkan nominal di bawah batas minimum lalu kirim donasi &
Nominal: Rp 5.000; Metode: QRIS &
Sistem menolak donasi dan menampilkan pesan bahwa minimum donasi Rp 10.000 &
Form menampilkan pesan kesalahan dan tidak membuat record payment baru &
\textcolor{green}{Berhasil} \\ \hline

9 &
Notifikasi Pembayaran &
Status pembayaran diperbarui dari Midtrans &
Selesaikan pembayaran di Midtrans lalu biarkan Midtrans mengirim notifikasi ke endpoint sistem &
Payload notifikasi Midtrans untuk transaksi dengan status \textit{settlement} &
Sistem memverifikasi signature, mengubah status payment menjadi \textit{paid/settlement}, dan menambah dana terkumpul kampanye &
Status payment berubah menjadi \textit{settlement} dan nilai \textit{raised\_amount} kampanye ikut bertambah &
\textcolor{green}{Berhasil} \\ \hline

\multicolumn{8}{|l|}{\textbf{D. Manajemen Profil dan Pengaturan}} \\ \hline

10 &
Update Profil &
Data profil pengguna berhasil diperbarui &
Login, buka halaman pengaturan profil, ubah nama, email, bio, dan simpan perubahan &
Nama baru, email baru yang unik, bio, dan nomor telepon &
Profil tersimpan dan informasi baru tampil di halaman profil publik dan pengaturan &
Profil tersimpan dengan data terbaru tanpa error validasi &
\textcolor{green}{Berhasil} \\ \hline

11 &
Ubah Password &
Password pengguna berhasil diubah &
Login, buka halaman ubah password, isi password saat ini dan password baru yang valid, lalu simpan &
Password saat ini benar; password baru minimal 8 karakter dan konfirmasi sama &
Password diubah, pengguna dapat login dengan password baru dan tidak lagi dengan password lama &
Password berhasil diubah; login dengan password baru berhasil dan password lama ditolak &
\textcolor{green}{Berhasil} \\ \hline

\multicolumn{8}{|l|}{\textbf{E. Manajemen Penarikan Dana}} \\ \hline

12 &
Request Penarikan &
Request penarikan diizinkan saat kampanye mencapai ≥ 80\% target &
Login sebagai pemilik kampanye yang sudah mencapai minimal 80\% target, buka halaman request penarikan dan isi form &
Kampanye dengan raised\_amount ≥ 80\% target; jumlah penarikan dan data rekening valid &
Sistem membuat request penarikan berstatus \textit{pending} dan dapat dilihat oleh admin &
Request penarikan berhasil dibuat dengan status \textit{pending} dan tersimpan di daftar penarikan &
\textcolor{green}{Berhasil} \\ \hline

13 &
Request Penarikan &
Request penarikan ditolak saat kampanye belum 80\% &
Login sebagai pemilik kampanye yang belum mencapai 80\% target, coba buka halaman request penarikan &
Kampanye dengan raised\_amount < 80\% target &
Sistem menolak pembuatan request penarikan dan menampilkan pesan bahwa syarat 80\% belum terpenuhi &
Pengguna tidak dapat membuat request penarikan dan mendapatkan pesan penolakan sesuai aturan bisnis &
\textcolor{green}{Berhasil} \\ \hline

\multicolumn{8}{|l|}{\textbf{F. Fitur Admin}} \\ \hline

14 &
Verifikasi Organisasi &
Admin menyetujui verifikasi organisasi &
Login sebagai admin, buka daftar verifikasi organisasi, pilih satu request \textit{pending}, klik \textit{Approve} &
Request verifikasi organisasi dengan dokumen valid dan status \textit{pending} &
Status verifikasi berubah menjadi \textit{approved} dan organisasi dapat digunakan pada kampanye &
Status request menjadi \textit{approved} dan organisasi tampil sebagai terverifikasi di form kampanye &
\textcolor{green}{Berhasil} \\ \hline

15 &
Verifikasi Organisasi &
Admin menolak verifikasi organisasi dengan alasan &
Login sebagai admin, pada daftar verifikasi pilih request \textit{pending}, klik \textit{Reject} dan isi alasan penolakan &
Request verifikasi organisasi yang belum lengkap; alasan penolakan: "Dokumen tidak lengkap" &
Status verifikasi berubah menjadi \textit{rejected} dan alasan penolakan tersimpan dan dapat dilihat pemilik &
Status berubah menjadi \textit{rejected} dan alasan penolakan muncul di tampilan pengguna &
\textcolor{green}{Berhasil} \\ \hline

16 &
Approve Penarikan &
Admin menyetujui request penarikan dana &
Login sebagai admin, buka daftar request penarikan berstatus \textit{pending}, klik \textit{Approve} &
Request penarikan valid dengan data rekening lengkap &
Status request berubah menjadi \textit{approved} sehingga pengguna dapat melanjutkan proses (mis. upload bukti) &
Status withdrawal berubah menjadi \textit{approved} dan terlihat di detail penarikan pengguna &
\textcolor{green}{Berhasil} \\ \hline

17 &
Verifikasi KTP &
Admin menyetujui verifikasi KTP pengguna &
Login sebagai admin, buka daftar verifikasi KTP \textit{pending}, tinjau dokumen, klik \textit{Approve} &
Request verifikasi KTP dengan foto dan data yang valid &
Status verifikasi KTP berubah menjadi \textit{approved} dan akun pengguna ditandai sebagai terverifikasi &
Status verifikasi KTP menjadi \textit{approved} dan badge/verifikasi tercatat pada akun pengguna &
\textcolor{green}{Berhasil} \\ \hline

18 &
Penghapusan Kampanye &
Admin menyetujui request penghapusan kampanye &
Login sebagai admin, buka daftar request hapus kampanye berstatus \textit{pending}, klik \textit{Approve} &
Request penghapusan kampanye yang diajukan pemilik &
Status request menjadi \textit{approved} sehingga pemilik dapat menghapus kampanye dari dashboard &
Status request penghapusan menjadi \textit{approved} dan kampanye dapat dihapus oleh pemiliknya &
\textcolor{green}{Berhasil} \\ \hline

\end{longtable}
\end{landscape}

\section{Hasil Penelitian}

\subsection{Ringkasan Hasil Testing}

Berdasarkan pengujian blackbox yang telah dilakukan pada platform EduSorong, berikut adalah ringkasan hasil penelitian:

\subsubsection{Statistik Testing}

\begin{table}[h]
\centering
\begin{tabular}{|l|c|c|}
\hline
\textbf{Kategori} & \textbf{Jumlah Test Case} & \textbf{Pass Rate} \\
\hline
Autentikasi & 3 & 100\% (3/3) \\
\hline
Manajemen Kampanye & 3 & 100\% (3/3) \\
\hline
Donasi dan Pembayaran & 3 & 100\% (3/3) \\
\hline
Manajemen Profil & 2 & 100\% (2/2) \\
\hline
Manajemen Penarikan Dana & 2 & 100\% (2/2) \\
\hline
Fungsi Admin & 5 & 100\% (5/5) \\
\hline
\textbf{Total} & \textbf{18} & \textbf{100\% (18/18)} \\
\hline
\end{tabular}
\caption{Statistik Hasil Testing}
\end{table}

\subsection{Temuan Utama}

\subsubsection{1. Fungsionalitas Autentikasi}
\begin{itemize}
    \item Sistem autentikasi berjalan dengan baik, termasuk validasi email unik dan panjang password
    \item Session management berfungsi dengan baik untuk login dan logout
    \item Error handling untuk kredensial yang salah sudah tepat
\end{itemize}

\subsubsection{2. Manajemen Kampanye}
\begin{itemize}
    \item Fitur CRUD kampanye berjalan dengan baik
    \item Validasi required field dan format data sudah tepat
    \item Integrasi dengan organisasi terverifikasi berfungsi dengan baik
    \item Workflow approval untuk penghapusan kampanye berjalan sesuai desain
\end{itemize}

\subsubsection{3. Sistem Donasi dan Pembayaran}
\begin{itemize}
    \item Integrasi dengan Midtrans untuk berbagai metode pembayaran (QRIS, E-Wallet, Virtual Account) berjalan dengan baik
    \item Validasi minimum donasi (Rp 10.000) berfungsi dengan baik
    \item Webhook handler untuk notifikasi pembayaran dari Midtrans berfungsi dengan baik
    \item Tracking status pembayaran berjalan dengan akurat
\end{itemize}

\subsubsection{4. Manajemen Profil}
\begin{itemize}
    \item Update profil berjalan dengan baik, termasuk validasi email unik (dengan ignore current user)
    \item Perubahan password dengan validasi current password berfungsi dengan baik
    \item Workflow verifikasi KTP berjalan sesuai desain
\end{itemize}

\subsubsection{5. Manajemen Penarikan Dana}
\begin{itemize}
    \item Validasi business rule 80\% target untuk penarikan dana berfungsi dengan baik
    \item Workflow approval penarikan dana berjalan dengan baik
    \item Upload dan verifikasi bukti penarikan berfungsi dengan baik
\end{itemize}

\subsubsection{6. Fungsi Admin}
\begin{itemize}
    \item Semua fungsi admin (approve/reject verifikasi, approve/reject penarikan, dll) berjalan dengan baik
    \item Admin dapat memberikan alasan untuk rejection yang jelas
    \item Workflow completion untuk withdrawal request berjalan dengan baik
\end{itemize}

\subsection{Kesimpulan}

Platform EduSorong telah berhasil melewati semua test case blackbox testing dengan tingkat keberhasilan 100\%. Semua fitur utama berfungsi sesuai dengan requirement dan use case yang telah ditetapkan. Sistem validasi, integrasi payment gateway, dan workflow approval berjalan dengan baik.

\subsection{Rekomendasi}

Meskipun semua test case berhasil, berikut beberapa rekomendasi untuk pengembangan selanjutnya:

\begin{enumerate}
    \item \textbf{Penambahan Test Case Negatif}: Disarankan untuk menambahkan lebih banyak test case negatif untuk edge cases yang lebih kompleks
    \item \textbf{Performance Testing}: Disarankan untuk melakukan performance testing, terutama untuk halaman yang menampilkan banyak data (daftar kampanye, feed donasi)
    \item \textbf{Security Testing}: Disarankan untuk melakukan security testing tambahan, terutama untuk validasi input dan proteksi terhadap SQL injection, XSS, dll
    \item \textbf{Integration Testing}: Disarankan untuk melakukan integration testing yang lebih mendalam dengan Midtrans, terutama untuk berbagai skenario pembayaran
    \item \textbf{User Acceptance Testing}: Disarankan untuk melakukan UAT dengan pengguna nyata untuk mendapatkan feedback dari perspektif end-user
\end{enumerate}

\section{Penutup}

Dokumen ini menyajikan hasil blackbox testing dan penelitian untuk platform EduSorong. Semua test case yang telah ditetapkan berhasil dilalui dengan baik, menunjukkan bahwa sistem telah memenuhi requirement fungsional yang ditetapkan. Platform siap untuk tahap pengujian selanjutnya atau deployment ke lingkungan production dengan catatan rekomendasi yang telah disebutkan.

\end{document}

